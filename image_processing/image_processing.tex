\documentclass[11pt]{article}

%packages
\usepackage{times}
\usepackage[small,compact]{titlesec}
\usepackage[small,it]{caption}
\usepackage[margin=1in]{geometry}
\usepackage{graphicx}
\usepackage{verbatim}
\usepackage{color}
\usepackage{hyperref}
\usepackage{amsmath}
\usepackage[final]{pdfpages}
\usepackage{setspace}
\usepackage{fancyhdr}
\usepackage{lipsum}

%header, footer, spacing setup
\pagestyle{fancyplain}
\singlespacing
\fancyhead{}
\fancyfoot{}
\fancyhead[CE,CO]{Processing Workshop -- Image Processing}
\fancyfoot[CE,CO]{\thepage}

\begin{document}

\centerline{\Large \bf Processing Workshop}
\medskip
\centerline{Course Notes by Nathan Lachenmyer}
%\medskip
%\centerline{Last Updated: \today}
\bigskip

\section{Introduction}
Processing can load, display, and modify images through the use of a data type called {\tt PImage}.  In the same way that {\tt int} or {\tt float} provides a data type to store numerical values, {\tt PImage} provides a structure for storing Using this data type, Processing can read {\tt .gif}, {\tt .jpg}, and {\tt .png} image files.  However, before you can display an image you must first make the image accessible to Processing.  This can be done by going to the Sketch menu of the Processing IDE and selecting ``Add File''.  You may then navigate to the image's location and select it to add it to the sketch's data folder.


\url{http://processing.org/learning/color/}

\section{PImage -- Basic Operations}
The first thing to do when interacting with a new data type is to learn about its constructors (which tell you how to create the data type) and methods (which allow you to modify the data once it has been created).  PImage is part of the standard Processing library; therefore you can find out more about it at \url{http://www.processing.org/}.  Go to \url{http://processing.org/reference/PImage.html} and read about the PImage data type, then answer the following questions:

\begin{enumerate}
\item 
\end{enumerate}




\end{document}
