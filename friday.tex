\documentclass[xcolor=dvipsnames]{beamer}	
\usetheme{Dresden}
\useoutertheme{infolines}
%Warsaw, Madrid, Dresden
\usecolortheme[named=Violet]{structure}
\setbeamertemplate{navigation symbols}{}
\usepackage[english]{babel}
\usepackage{times}
\usepackage[T1]{fontenc}
\usepackage{amsmath}
\usepackage{graphicx}
\usepackage{pgfpages}
\setbeamertemplate{footline}[frame number]

\title{Processing Workshop}
\subtitle{Friday Night Edition}
\author{Nathan Lachenmyer}
\institute{CEMI Electronic Media Institute}
\date{\today}

\begin{document}

\begin{frame}
  \titlepage
\end{frame}

\begin{frame}
\frametitle{What is Processing?}
\begin{center}
Processing is a programming language specifically designed to \textbf{generate} and \textbf{modify} images.
\end{center}
\end{frame}

\begin{frame}
\frametitle{What is Processing?}
\begin{center}
Processing is a software sketchbook -- it makes it easy to \textbf{explore} and \textbf{refine} ideas \textbf{quickly}.
\end{center}
\end{frame}

\begin{frame}
\frametitle{What is Processing?}
\begin{center}
Processing was designed to engage people with \textbf{visual} and \textbf{spatial} minds, to open up programming to \textbf{artists} and \textbf{designers}.
\end{center}
\end{frame}

\begin{frame}
\frametitle{About Me}
Technical Background: Physics, Electrical Engineering \\
Need image here!
\end{frame}

\begin{frame}
\frametitle{About Me}
\begin{center}
Turn \textbf{science} into \textbf{art}.
\end{center}
\end{frame}

\begin{frame}
\frametitle{My Work}
\framesubtitle{Randomness/Complexity}
\begin{center}
\includegraphics[width=0.8\linewidth]{/home/scottnla/Dropbox/processing/particleSystems/vectorFields/perlin/perlin-1581.png}
\end{center}
\end{frame}

\begin{frame}
\frametitle{My Work}
\framesubtitle{Geometry}
\begin{center}
\includegraphics[width=0.8\linewidth]{/home/scottnla/Dropbox/processing/particleSystems/kaleidoscopes/perlinKaleidoscope/perlinKaleidoscope-2859.png}
\end{center}
\end{frame}

\begin{frame}
\frametitle{My Work}
\framesubtitle{Physical Phenomena}
\begin{center}
\includegraphics[width=0.7\linewidth]{/home/scottnla/Dropbox/processing/diffusion2/diffuse-react-53726.png}
\end{center}
\end{frame}

\begin{frame}
\frametitle{My Work}
\framesubtitle{Physical Phenomena}
\begin{itemize}
\item Name
\item What do you want to get out of this class?
\item 
\end{itemize}
\end{frame}

\begin{frame}
\frametitle{The Plan}
\framesubtitle{Friday}
\begin{itemize}
\item Install Processing
\item Download Course Materials
\item Brief introduction to programming concepts
\item Practice!
\end{itemize}
\end{frame}

\begin{frame}
\frametitle{The Plan}
\framesubtitle{Friday}
\begin{itemize}
\item 10:00 -- Control/Logic
\item 12:00 -- Lunch Break
\item 13:00 -- Project 1
\item 15:00 -- Project 2
\end{itemize}
\end{frame}

\begin{frame}
\frametitle{Install Processing}
\begin{center}
\url{http://www.processing.org/download} \\
Download Processing 1.5.1 (\textbf{NOT} 2.0b3!)
\end{center}
\end{frame}

\begin{frame}
\frametitle{Environment}
%picture of processing environment
\end{frame}

%piece of demo code

%fundamental objects -- functions and expressions

\begin{frame}
\frametitle{Structure}
Human language is flexible in terms of diction and syntax.
%example?
\end{frame}

\begin{frame}
\frametitle{Structure}
Computer language is not.
%example?
\end{frame}

\begin{frame}
\frametitle{Structure}
Expression $\approx$ Phrase
\end{frame}
%example

\begin{frame}
\frametitle{Structure}
Statement $\approx$ Sentence
\end{frame}
%example
%semicolon ~ period!

\begin{frame}
\frametitle{Structure}
Functions $\approx$ Paragraphs
\end{frame}
%example

\begin{frame}
\frametitle{Structure}
Capitalization MATTERS
White Space DOES NOT
// creates a comment
\end{frame}

\begin{frame}
\frametitle{Primitives}
\end{frame}

\begin{frame}
\frametitle{Primitives}
Coordinate System %image
\end{frame}

\begin{frame}
\frametitle{Primitives}
point
\end{frame}

\begin{frame}
\frametitle{Primitives}
line
\end{frame}

\begin{frame}
\frametitle{Primitives}
ellipse
\end{frame}

\begin{frame}
\frametitle{Primitives}

\end{frame}

\begin{frame}
\frametitle{Primitives}
rectangle
\end{frame}

\begin{frame}
\frametitle{Primitives}
triangle
\end{frame}

%draw some shapes!

%try changing window size -- wouldn't it be nice if objects resized automatically?

%introduce variables --> need to know data types

\begin{frame}
\frametitle{Data Types}
\end{frame}

%\begin{frame}
%\frametitle{Functions}
%\end{frame}

\end{document}
%http://www.openprocessing.org/classroom/2043
